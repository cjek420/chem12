\documentclass[fleqn]{exam}
\usepackage{chemfig}
\usepackage{siunitx}


\begin{document}

\begin{questions}
  \question
  5.0 g sample of magnesium reacts completely with a hydrochloric acid solution after 150 s. Express the average rate of consumption of magnesium,in units of g/min.
  \vspace{1in}
  \question How long will it take to completely react 45.0 g of \chemfig{CaCO_3}(s) with dilute HCl(aq) if the reaction proceeds at an average rate of 2.35 g \chemfig{CaCO_3}(s)/min under a given set of conditions?
  \vspace{1in}
  \question The electrolysis of water produces oxygen gas at the rate of 32.5 mL/min in a certain experiment. What volume of oxygen gas can be produced in 7.50 min?
  \vspace{1in}
  \question Which of the following are acceptable units for expressing reaction rate?
  \begin{parts}
    \part moles/second (c) (moles/litre)/second (e) millilitres/hour
    \vspace{1in}
    \part minutes/metre (d) grams/litre (f) grams/minute
    \vspace{1in}
  \end{parts}

  \question Hydrogen and oxygen gas react in a fuel cell to produce water according to the equation:
  \newline
  \schemestart
  2 \chemfig {H_2}(g) \+ \chemfig{O_2}(g) \arrow{->} 2 \chemfig{H_2O}(l).
  \schemestop
  \newline
  If the rate water production is 1.34 mol/min, what is the rate of oxygen gas consumption expressed in mol/min?
  \vspace{1in}

  \newpage

  \question
  The following data was obtained for the above reaction (mass includes beaker and contents).


  Plot the above data on the graph below.

  \vspace{2in}

  Now answer the following questions.

  \begin{parts}
    \part Why is the mass decreasing?
    \vspace{1in}
    \part What is the slope of the line in the above graph, using: slope = RISE/RUN?
    \vspace{1in}
    \part What are the units of: (i) the RISE? (ii) the RUN? (iii) the slope?
    \vspace{1in}
    \part What units would you expect to use for the rate of this reaction?
    \vspace{1in}
    \part What relationship exists between the slope of the graph and the rate of the reaction? State the value found for the experimentally-determined reaction rate.
    \vspace{1in}
  \end{parts}

  \question When measuring the rate at which the mass of copper metal decreases during a reaction with nitric
  acid, why can't you just put the reaction vessel on a digital balance and record the decrease in mass as
  the copper is used up?
  \vspace{1in}

  \question
  \begin{parts}
    \part Solutions of Cu?*(aq) are blue, while solutions of Ag*(aq) are colourless. Use only this
    information to describe how you would measure the rate of the reaction:
    \newline
    \schemestart
    \chemfig{2 Ag^+}(aq) + Cu(s) \arrow{->} 2 Ag(s) + \chemfig{Cu^{2+}}(aq) + 35 kJ.
    \schemestop
    \newline
    \vspace{1in}
    \part Suggest two more methods that could be used to determine the rate of the reaction in part (a).
    For each method, state the property that you are monitoring.
    \vspace{1in}
  \end{parts}
  \question
  \begin{parts}
    \part You are to measure the rate of the reaction:
    \newline
    \schemestart
    \chemfig{H_2}(g) + \chemfig{Cl_2}(g) \arrow{->} 2 HCl(g).
    \schemestop
    \newline
    Why is gas pressure
    NOT a good property to monitor in order to determine the reaction rate?
    \vspace{1in}
    \part Calculate the reaction rate, in mol HCl/s, if 1.2 g of HCl(g) are produced in 2.0 min.
    \vspace{1in}
    \part If the rate of consumption of hydrogen gas under certain conditions is 0.200 L/min, what is the
    rate of production of HCl(g)?
    \vspace{1in}
  \end{parts}

\end{questions}

\end{document}
