\documentclass[fleqn]{exam}
\usepackage{chemfig}
\usepackage{amssymb}
\usepackage{siunitx}

\begin{document}

\begin{questions}
  \setcounter{question}{28}

  \question 
  The reaction 
  \schemestart 
  \chemfig{C_2H_4}(g) + \chemfig{Br_2}(g) \arrow{->} \chemfig{C_2H_4Br_2}(g)
  \schemestop 
  proceeds very fast at room temperature.
  \begin{parts}
    \part Which of the following KE diagrams would best explain the rate of this reaction? ("ME" is the
    minimum KE required before a molecule can react.)
    \vspace{1in}
    \part If the temperature were increased by 10°C, would the reaction rate double? Explain.
    \vspace{1in}
  \end{parts}
  \question What happens to the shape of the KE distribution curve if the:
  \begin{parts}
    \part reactant is used up at a constant temperature?
    \vspace{1in}
    \part temperature is decreased?
    \vspace{1in}

    \part reactant surface area is increased?
    \vspace{1in}

    \part concentration of reactants is increased
    \vspace{1in}
  \end{parts}

  \question The initial rate of consumption of A in the reaction A $\rightarrow$  B is very slow: \num{1.0d7} molls at 20°C.
  Estimate the rate of the reaction at 40°C.
  \vspace{1in}
  \question If the rate of a slow reaction is \num{2.0d-4} mol/s at 10°C, estimate the rate at 40°C.
  \vspace{1in}
  \question Why don't the oceans convert to nitric acid?
  \vspace{1in}
  \question
  \begin{parts}
    \part Draw a PE diagram for a fast exothermic reaction.
    \vspace{1in}
    \part Draw a PE diagram for a slow exothermic reaction.
    \vspace{1in}
    \part Draw a PE diagram for a fast endothermic reaction.
    \vspace{1in}
    \part Draw a PE diagram for a slow endothermic reaction.
    \vspace{1in}
    \part How is the size of the "energy hill" related to the number of molecules which have sufficient KE to
    pass over the top of the hill?
  \end{parts}
  \vspace{1in}
  \question If two reactant molecules collide with sufficient KE, are they guaranteed to have collision?
  \vspace{1in}
  \question
  \begin{parts}
    \part As two reactant particles approach each other, what happens to
    (i) their KE? Why? (ii) their PE? Why?
    \vspace{1in}
    \part The total energy of a system is given by: E TOTAL = PE + KE. How does the value of E TOTAL
    before a collision compare to the value of E TOTAL after a collision?
  \end{parts}
  \vspace{1in}
  \question The following is a PE diagram for a collision between molecules A2 and B2. The molecules collide with
  favorable geometry.

  (a) if \chemfig{A_2} and \chemfig{B_2} had collided with less favourable geometry how to that shown above?
  (b) Why does PE decrease when going from the activated complex to the products, AB?
  (c) Is the overall reaction exothermic or endothermic?
  (d) Write a balanced equation for the reaction, including the value for the enthalpy.
  (e) What is the value of the activation energy in the above reaction?
  \vspace{1in}
  \question The bond energies
  of F2 and of l2 are almost identical. Would you expect the activation energy for
  \schemestart
  \chemfig{H_2 + F_2} \arrow{->} 2 HF
  \schemestop
  to be equal to, greater than, or less than the activation energy for
  \schemestart
  \chemfig{H_2+l_2} \arrow{->}2 Hl?
  \schemestop
  [Hint: why does an activation energy exist in the first place?]

  \vspace{1in}
  \question Carbon exists in two forms, or ALLOTROPES, called graphite and diamond. The enthalpy for yet one can't simply heat black, opaque and reaction converting graphite to diamond is only 2 kJ, inexpensive graphite and turn it into transparent and precious diamond. Suggest a reason why the reaction is so difficult to carry out.
  \vspace{1in}
  \question After a reaction, the product molecules have less kinetic energy than the original reactant molecules.
  Is the reaction endothermic or exothermic? Explain your answer.
  \vspace{1in}
  \question If $\Delta$H =-15 kJ and Ea(f) = 40 kJ, what is the value of Ea(r)?
  \vspace{1in}
  \question A reaction has Ea(f) = 55 kJ and Ea(r) = 30 kJ. Is the reaction exothermic or endothermia?
  \vspace{1in}
  \question Draw and label a PE diagram for the reaction:
  \schemestart
  2 NOBr(g) \arrow{->} 2 NO(g) + \chemfig{Br_2}(g) + 50 kJ
  \schemestop

  in which Ea(f) = 30 kJ. Indicate on your diagram the point at which the activated complex exists.

  \vspace{1in}
  \question Draw and label a PE diagram to show the enthalpy change and activation energies for a reaction in
  which: R + 25 kJ $\rightarrow$ P and Ea(r) = 10 kJ.

  \vspace{1in}
  \question Draw and label a PE diagram showing the enthalpy change and activation energies for a reaction in
  which Ea(f) = 20 kJ and Ea(r) = 45 kJ.



\end{questions}

\end{document}
