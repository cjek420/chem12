\documentclass[fleqn]{exam}
\usepackage{chemfig}
\usepackage{siunitx}

\begin{document}

\begin{questions}
  \setcounter{question}{45}

  \question
  A reaction is thought to proceed according to the following mechanism.
  \newline
  \schemestart
  \chemfig{2 NO + H_2} \arrow{->}  \chemfig{N_2 + H_2O_2} (slow)
  \schemestop

  \schemestart
  \chemfig{H_2O_2 + H_2} \arrow{->} \chemfig{2 H_2O} (fast)
  \schemestop

  \begin{parts}
    \part What is the overall reaction equation?
    \vspace{0.5in}
    \part What is true about the [\chemfig{H_2O_2}] at any time during the reaction?
    \vspace{0.5in}
    \part Which of the steps in the mechanism is the rate-determining step?
    \vspace{0.5in}
    \part What would happen to the overall rate it some extra NO was injected into reaction mixture?
    \vspace{0.5in}
    \part If it were somehow possible to speed up the second step in the mechanism, what effect would this have on the overall rate of the reaction?
    \vspace{0.5in}
    \part What is the formula of the activated complex in the 1st step of the reaction? In the second step?
    \vspace{0.5in}
    \part How many elemetary process are involved in the reaction?
    \vspace{0.5in}
  \end{parts}

  \question  What is the difference between an activated complex and a reaction intermediate?
  \vspace{1in}

  \question The reaction A $\rightarrow$ C is known to have the mechanism:
  \newline
  A $\rightarrow$ B (fast)
  \newline
  B $\rightarrow$ C (slow)
  \newline
  What would you expect to be true about the concentration of B as the reaction proceeds?
  \vspace{1in}

  \question You have been told that phosphorous can be prepared by means of the reaction
  \newline
  \schemestart
  \chemfig{2 Ca_3{(}PO_4{)}_2 + 6 SiO_2 + 10 C} \arrow{->} \chemfig{P_4 + 6 CaSiO_3 + 10 CO}.
  \schemestop
  \newline
  Why can you be certain that the reaction equation shown does not represent a reaction mechanism?
  \vspace{1in}

  \question A two step mechanism IS proposed for a reaction:
  \schemestart
  \chemfig{ClO^{-} + ClO^{-}} \arrow{->} \chemfig{ClO^{-}_2, + Cl^{-}}
  \schemestop
  \newline
  \schemestart
  \chemfig{ClO^{-}_2, + ClO^{-}} \arrow{->} \chemfig{ClO^{-}_3, + Cl^{-}}
  \schemestop
  \begin{parts}
    \part What is the overall reaction which occurs?
    \vspace{0.5in}
    \part Is ClO, a reaction intermediate or an activated complex?
    \vspace{0.5in}
    \part What is the chemical formula for the activated complex in the second step?
    \vspace{0.5in}
  \end{parts}

  \question The decomposition of acetone, \chemfig{{(}CH_3{)}_2CO}, proceeds according to
  \newline
  \schemestart
  \chemfig{2{(}CH_3{)}_2CO} \arrow{->} \chemfig{C_2H_4 + 2 CO + 2 CH_4}.
  \schemestop
  \newline
  If the decomposition is a two-step reaction, and the second step is
  \newline
  \schemestart
  \chemfig{2 CH_2CO} \arrow{->} \chemfig{C_2H_4 + 2 CO},
  \schemestop
  \begin{parts}
    \part what is the first step?
    \vspace{0.5in}
    \part what is the formula for the activated complex in the first step? The second step?
  \end{parts}
  \vspace{1in}

  \question A chemist suggested that the reaction:
  \schemestart
  \chemfig{2 NO + O_2} \arrow{->} \chemfig{NO_2} has a three-step mechanism.
  \schemestop
  \newline
  If the proposed first and third steps are:
  \newline
  \schemestart
  \chemfig{2 NO} \arrow{->} \chemfig{N_2O_2} (first)
  \schemestop
  \newline
  \schemestart
  \chemfig{N_2O_4} \arrow{->} \chemfig{2 NO_2} (third),
  \schemestop
  \begin{parts}
    \part what is the second step in the proposed reaction?
    \vspace{0.5in}
    \part what is the formula of the activated complex in the second step?
    \vspace{0.5in}
  \end{parts}

  \question The reaction between gaseous hydrogen and chlorine proceeds as follows.
  \newline
  \schemestart
  \chemfig{Cl_2} + light \arrow{->} 2 Cl .... (1)
  \schemestop
  \newline
  \schemestart
  \chemfig{Cl+H_2} \arrow{->} \chemfig{HCl + H} .... (2)
  \schemestop
  \newline
  \schemestart
  \chemfig{H+ Cl_2} \arrow{->} \chemfig{HCl + Cl} .... (3)
  \schemestop
  \begin{parts}
    \part Suggest what step might occur after step 3? [Hint: Steps 2 and 3 show what happens when an individual pair of Cl and \chemfig{H_2} react; not all the Cl's and \chemfig{H_2}'s react at once.]
    \vspace{0.5in}
    \part What function is served by the light?
    \vspace{0.5in}
    \part Suggest why this reaction is called a "chain reaction".
    \vspace{0.5in}
  \end{parts}

  \question 54. Which of the steps in the reaction
  \schemestart
  \chemfig{4 HBr + O_2} \arrow{->} \chemfig{2 H_2O + 2 Br_2}
  \schemestop
  \newline
  has the greatest activation energy?
  Which has the least?
  \vspace{1in}

  \question In the following PE diagram :
  \begin{parts}
    \part How many steps does this reaction have?
    \vspace{0.5in}
    \part Is the second step (B $\rightarrow$ C) exothermic or endothermic?
    \vspace{0.5in}
    \part Is the overall reaction exothermic or endothermic?
  \end{parts}

\end{questions}

\end{document}
