\documentclass[fleqn]{exam}
\usepackage{chemfig}
\usepackage{siunitx}

\begin{document}

\begin{questions}
  \setcounter{question}{55}

  \question In the following reaction mechanisms identify
  (i) the catalyst
  (ii) the reaction intermediate(s)
  (iii) the overall reaction
  \begin{parts}
    \part
    \schemestart
    \chemfig{CH_2 = CH_2 + H^+} \arrow{->} \chemfig{CH_3 - CH_2^+}
    \schemestop
    \newline
    \schemestart
    \chemfig{CH_3 - CH_2^+ + H_2O} \arrow{->} \chemfig{CH_3 - CH_2 - OH + H^+}
    \schemestop
    \vspace{0.5in}
    \part
    A + B $\rightarrow$ C
    \newline
    C + D $\rightarrow$ E + F
    \newline
    E + B $\rightarrow$ D + F
    \vspace{0.5in}
    \part
    \schemestart
    \chemfig{NH_2NO_2 + CH_3COO^{-}} \arrow{->} \chemfig{CH_3COOH + NHNO_2^{-}}
    \schemestop
    \newline
    \schemestart
    \chemfig{NHNO_2^{-}} \arrow{->} \chemfig{N_2O+OH^{-}}
    \schemestop
    \newline
    \schemestart
    \chemfig{OH^{-} + CH_3COOH} \arrow{->} \chemfig{H_2O + CH_3COO^{-}}
    \schemestop
    \vspace{0.5in}
    \part
    \schemestart
    \chemfig{Pt + C_2H_2} \arrow{->} \chemfig{PtC_2H_2}
    \schemestop
    \newline
    \schemestart
    \chemfig{PtC_2H_2 + H_2} \arrow{->} \chemfig{PtC_2H_4}
    \schemestop
    \newline
    \schemestart
    \chemfig{PtC_2H_4 + H_2} \arrow{->} \chemfig{Pt + C_2H_6}
    \schemestop
    \vspace{0.5in}
    \part A $\rightarrow$ 2B
    \newline
    B + C $\rightarrow$ D + E
    \newline
    D + F $\rightarrow$ C + G
    \newline
    B + G $\rightarrow$ H
    \vspace{0.5in}
  \end{parts}

  \question ``All catalyzed reaction mechanisms have more than one step.'' Why must this statement be true?
  \vspace{1in}
  \question Suppose a catalyzed reaction is occurring in a reaction container. If the catalyst is removed, will the reaction stop completely? Explain your answer.
  \vspace{1in}
  \question Can a catalyst cause an exothermic reaction to become endothermic, or vice versa? Explain.
  \vspace{1in}
  \question Consider the following reaction mechanism: X+Y$\rightarrow$ Z (very fast) \newline
  Z + Y $\rightarrow$ P (very fast) \newline
  P + Y $\rightarrow$ Q (slow) \newline
  Suppose there was a catalyst that worked on step 1, and another catalyst that worked on step 3.
  Which catalyst would be ineffective in increasing the rate of the overall reaction?
  \vspace{1in}
  \question If you have a slow reaction and add a substance that provides an alternate reaction mechanism having a higher activation energy, what will happen to the reaction rate? Why does this occur?
  \vspace{1in}
  \question Chlorine atoms are present in the upper atmosphere as a result of emissions from volcanoes and man-made pollutants. The reaction between chlorine atoms and ozone is thought to proceed by a 2-step mechanism:
  \schemestart
  \chemfig{O_3 + Cl} \arrow{->} \chemfig{ClO + O_2}
  \schemestop
  \newline
  \schemestart
  \chemfig{O + ClO} \arrow{->} \chemfig{O_2 + Cl}
  \schemestop
  \newline
  \begin{parts}
    \part What is the overall reaction?
    \vspace{1in}
    \part Identify any reaction intermediates or catalysts present.
    \vspace{1in}
    \part Suggest a reason why chlorine atoms in the upper atmosphere are a threat to the environment.
    \vspace{1in}
    \part Why does the presence of chlorine atoms in the upper atmosphere allow more UV light to reach the earth's surface? \[Hint: look at reaction (3) on the previous page.\]
  \end{parts}
  \newpage
  \question The catalyzed reaction between \chemfig{CH_3OH} (an alcohol) and \chemfig{CH_3COOH} (an organic acid) to make \chemfig{CH_3COOCH_3} (an ester) proceeds as follows.
  \newline
  \schemestart
  \chemfig{CH_3OH + H^+} \arrow{->}  \chemfig{CH_3OH_2^2}
  \schemestop
  \newline
  \schemestart
  \chemfig{CH_3OH_2^2} \arrow{->}  \chemfig{CH_3^+ + H_2O}
  \schemestop
  \newline
  \schemestart
  \chemfig{CH_3^+ + CH_3COOH} \arrow{->}  \chemfig{CH_3COOHCH_3^+}
  \schemestop
  \newline
  \schemestart
  \chemfig{CH_3COOHCH_3^+} \arrow{->}  \chemfig{CH_3COOCH_3 + H^+}
  \schemestop
  \newline
  \begin{parts}
    \part What is the overall reaction?
    \vspace{1in}
    \part Why is the reaction said to be "acid catalyzed"?
    \vspace{1in}
    \part If the \chemfig{H^+} used in the first step was radioactive, Would the \chemfig{CH_3COOCH_3} produced in the 4th step contain a radioactive hydrogen atom? Why?
  \end{parts}

\end{questions}

\end{document}
