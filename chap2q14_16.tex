\documentclass[fleqn]{exam}
\usepackage{chemfig}
\usepackage{siunitx}

\begin{document}

\begin{questions}
  \setcounter{question}{13}

  \question In each of the following pairs of substances, select the one which has greater entropy.
  \begin{parts}
    \part \chemfig{H_2O}(l) or \chemfig{H_2O}(g)
    \vspace{.25in}
    \part \chemfig{Cl_2}(g) or 2 \chemfig{Cl^{-}}(aq)
    \vspace{.25in}
    \part \chemfig{NH_3}(l) or \chemfig{NH_3}(aq)
    \vspace{.25in}
    \part \chemfig{CH_3COOH}(aq) or \chemfig{CH_3COO^{-}}(aq) + \chemfig{H^+}(aq)
    \vspace{.25in}
  \end{parts}

  \question In each of the following, decide
  \newline
  i) which side is favoured by the tendency to minimum enthalpy; that is, which side of the reaction the lower energy.
  \newline
  ii) which side is favoured by the tendency to maximum entropy; that is, which side of the reaction the more random species.
  \newline
  iii) whether the reaction will be
  \newline
  a spontaneous reaction which goes to completion (``GOES 100\%''), or
  \newline
  a non-spontaneous reaction in which NO products are formed (``WON'T OCCUR''), or
  \newline
  a spontaneous equilibrium reaction in which the tendency to minimum enthalpy will be balanced by an opposing tendency to maximum entropy (''EQUILIBRIUM'').
  \newline
  Note: in parts (a) to (d) all the species are GASES

  \begin{parts}
    \part see graph
    \vspace{1in}
    \part see graph
    \vspace{1in}
    \part see graph
    \vspace{1in}
    \part see graph
    \vspace{1in}
    \part  \schemestart \chemfig{H_2S0_4}(l) + \chemfig{H_2O}(l) \arrow{->} + \chemfig{H_2SO_4}\schemestop (aq) + 150 kJ
    \vspace{1in}
    \part \schemestart \chemfig{C_2H_6}(g) \arrow{->} \chemfig{C_2H_2}(g) + \chemfig{2 H_2}\schemestop (g);  $\Delta$H = 311 kJ
    \vspace{1in}
    \part \schemestart \chemfig{C_2H_2}(g) + \chemfig{Ca{(}OH{)}_2}(aq) \arrow{->} \chemfig{CaC_2}(s) + \chemfig{2H_20}(l)\schemestop (g); $\Delta$H = 183 kJ
    \vspace{1in}
    \part \schemestart \chemfig{2C{(}s{)} + O_2}(g) \arrow{->} 2 CO\schemestop (g); $\Delta$H =-221 kJ
  \end{parts}
  \newpage
  \question What tendencies to minimum enthalpy and maximum entropy must exist in the following situations?
  \begin{parts}
    \vspace{1in}
    \part Liquid nitroglycerine explodes, forming an expanding cloud of gases.
    \vspace{1in}
    \part Solid AgBr is almost insoluble in water; that is, very little \chemfig{Ag^+}(aq) and \chemfig{Br^{-}}(aq) are formed when AgBr(s) is mixed with water.
    \vspace{1in}
    \part Water and alcohol mix completely in any proportions; that is, they are ``miscible''.
    \vspace{1in}
    \part The reaction:  \schemestart\chemfig{3 N_2}(g) + Pb(s) \arrow{->} \chemfig{Pb{(}N_3{)}_2}\schemestop(s) does not occur.
    \vspace{1in}
    \part When \chemfig{N_2O_4}(g) is put in a container, some of it decomposes into \chemfig{2 NO_2}(g).
    \vspace{1in}
    \part Smoke, carbon dioxide and water vapour will not react to make wood and oxygen.
  \end{parts}


\end{questions}

\end{document}
