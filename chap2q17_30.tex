\documentclass[fleqn]{exam}
\usepackage{chemfig}
\usepackage{siunitx}

\begin{document}

\begin{questions}
  \setcounter{question}{16}

  \question The equilibrium is: \schemestart \chemfig{N_2O_3}(g) \arrow {<=>} NO(g) + \chemfig{NO_2} \schemestop (g)
  \begin{parts}
    \part increase the [NO]
    \vspace{1in}
    \part increase the [\chemfig{N_2O_3}]
    \vspace{1in}
    \part increase the pressure by decreasing the volume
    \vspace{1in}
    \part add a catalyst
    \vspace{1in}
  \end{parts}

  \question
  The equilibrium is: \schemestart \chemfig{2 H_2}(g) + 2 NO(g) \arrow{<=>} \chemfig{N_2}(g) + 2 \chemfig{H_20}(g)\schemestop
  \begin{parts}
    \part decrease the [\chemfig{N_2}]
    \vspace{1in}
    \part decrease the [NO]
    \vspace{1in}
    \part decrease the pressure by increasing the volume
    \vspace{1in}
  \end{parts}

  \question The equilibrium is: \schemestart 2 CO(g) + \chemfig{O_2}(g) \arrow{<=>} 2 \chemfig{CO_2}\schemestop(g) + 566 kJ
  \begin{parts}
    \part increase the temperature
    \vspace{1in}
    \part increase the [O2]
    \vspace{1in}
    \part introduce a catalyst
    \vspace{1in}
  \end{parts}

  \question The equilibrium is: \schemestart \chemfig{l_2}(g) + \chemfig{Cl_2}(g) \arrow{<=>} 2 lCl\schemestop(g); $\Delta$H = 35.0 kJ.
  \begin{parts}
    \part decrease the temperature
    \vspace{1in}
    \part decrease the [\chemfig{Cl_2}]
    \vspace{1in}
    \part increase the pressure by decreasing the volume
    \vspace{1in}
  \end{parts}

  \newpage
  For each of Exercises 21 23, describe the effect on the concentration of the bold substance by following changes. Write INC for increase, DEC for decrease or NC for no change.
  \vspace{.25in}
  \question The equilibrium is: \schemestart \chemfig{N_2}(g) + 3 \chemfig{H_2}(g) \arrow{<=>} 2 \chemfig{NH_3}(g)\schemestop ; $\Delta$H = -92 kJ.
  \begin{parts}
    \part increase the [\chemfig{N_2}]
    \vspace{1in}
    \part increase the temperature
    \vspace{1in}
    \part increase the volume
    \vspace{1in}
    \part add a catalyst
    \vspace{1in}
  \end{parts}

  \question The equilibrium is: \schemestart \chemfig{2 HF} (g) \arrow{<=>} \chemfig{F_2}(g) + \chemfig{H_2}(g)\schemestop ; $\Delta$H = 536 kJ.
  \begin{parts}
    \part decrease the temperature
    \vspace{1in}
    \part increase the [\chemfig{H_2}]
    \vspace{1in}
    \part increase the volume
    \vspace{1in}
  \end{parts}

  \question equilibrium is: \schemestart \chemfig{SnO_2}(s) + 2 CO(g) \arrow{<=>} Sn(s) + 2 \chemfig{CO_2}\schemestop(g); $\Delta$H= 13 kJ.
  \begin{parts}
    \part increase the temperature
    \vspace{1in}
    \part add a catalyst
    \vspace{1in}
    \part increase the [CO]
    \vspace{1in}
  \end{parts}

  NOTE: In Exercises 24-26 the relative positioning of the molecules is not relevant; simply place them on
  the graph so the reactants are separated from the products. The Only thing required here is to show
  what an individual substance's concentration does after the conditions change.
  \vspace{.25in}

  \question The equilibrium is: \schemestart \chemfig{H_2}(g) + \chemfig{l_2}(g) \arrow{<=>} \schemestop 2 Hl(g) + 52 kJ
  \begin{parts}
    \part increase the temperature
    \vspace{1in}
    \part inject some \chemfig{H_2}(g)
    \vspace{1in}
    \part decrease the volume
    \vspace{1in}
    \part add a catalyst
    \vspace{1in}
  \end{parts}

  \question The equilibrium is: \schemestart \chemfig{2 SO_2}(g) + \chemfig{O_2}(g) \arrow{<=>} 2 \chemfig{SO_3}\schemestop(g); $\Delta$H = -197 kJ.
  \begin{parts}
    \part inject some \chemfig{SO_2}(g)
    \vspace{1in}
    \part increase the volume
    \vspace{1in}
    \part decrease the temperature
    \vspace{1in}
    \part increase the [\chemfig{SO_3}]
    \vspace{1in}
  \end{parts}

  \newpage
  \question The equilibrium is: \schemestart \chemfig{CO}(g) + \chemfig{H_2O}(g) \arrow{<=>} \chemfig{CO_2}(g) + \chemfig{H_2}\schemestop(g); $\Delta$H = -41 kJ.
  \begin{parts}
    \part inject some \chemfig{CO_2}(g)
    \vspace{1in}
    \part remove some of the \chemfig{H_2O}(g) with a very rapidly acting drying agent
    \vspace{1in}
    \part increase the temperature
    \vspace{1in}
    \part decrease the pressure by increasing the volume
    \vspace{1in}
  \end{parts}

  Interpret the following graphs in terms of the changes which must have been imposed on the equilibrium.

  \question The equilibrium is: \schemestart \chemfig{PCl_5}(g) + 92.5 kJ \arrow{<=>} \chemfig{PCl_3}(g) + \chemfig{Cl_2}(g)\schemestop

  \vspace{2in}


  \question The equilibrium is: \schemestart \chemfig{H_2O}(g) + \chemfig{Cl_2O}(g) \arrow{<=>} 2 HOCl(g) + 70 kJ\schemestop


  \vspace{2in}

  \question \schemestart \chemfig{N_2}(g) + 3 \chemfig{H_2}(g) \arrow{<=>} \chemfig{2 NH_3}(g) \schemestop + 92 kJ.
  \begin{parts}
    \part In order to get highest yield of \chemfig{NH_3}(g), should you use high or low pressure?
    \vspace{.5in}
    \part In order to get the the highest yield of \chemfig{NH_3}(g), should you use high or low temperature?
    \vspace{.5in}
    \part In order to have the fastest reaction rate, should you use high or low temperature?
    \vspace{.5in}
    \part Look at your answers to parts (b) and (c). What problem now exists? Suggest a suitable way to resolve this problem.
    \vspace{.5in}
    \part What else can be done to speed up the reaction rate? (Industry uses iron oxide for this purpose, in the form of ground up, rusted automobile bodies.)
    \vspace{.5in}
  \end{parts}

  \question \schemestart \chemfig{CaCO_3}(s) + 175 kJ \arrow{<=>} CaO(s) + \chemfig{CO_2}\schemestop(g)
  \begin{parts}
    \part Should high or low temperatures be used to get the greatest yield of CaO?
    \vspace{.5in}
    \part Should high or low pressures be used to get the highest yield of CaO? How would you accomplish this in actual practice?
    \vspace{.5in}
    \part Should high or low temperatures be used to obtain the fastest reaction rate? Is there a conflict between the answers for parts (a) and (c)?
    \vspace{.5in}
  \end{parts}


\end{questions}

\end{document}
