\documentclass[fleqn]{exam}
\usepackage{chemfig}
\usepackage{siunitx}

\begin{document}

\begin{questions}

  \question Strictly speaking, NO system can be completely closed for example, hot liquid in the best thermos eventually cools down but for most purposes a system can effectively be "closed". How would you create a system which is more or less closed with respect to:
  \begin{parts}
    \part heat loss?
    \vspace{.75in}
    \part light?
    \vspace{.75in}
    \part loss of mass?
    \vspace{.75in}
  \end{parts}
  \question Read the following observations and then answer the questions. \newline
  Two sealed glass tubes containing a mixture of a red-brown gas, \chemfig{NO_2}(g), and a colourless gas, \chemfig{N_20_4}(g), are observed. The colour is an identical medium red-brown in each tube and there is no visible change in the colour of the contents as time passes.
  \newline
  One tube is placed in a beaker of boiling water for a minute. The contents of the tube become much darker red-brown in colour. Upon first placing the tube in the hot water, the colour gets continually darker, but after a few seconds the colour stops changing.
  \newline
  The second tube is placed in a beaker containing dry ice at \qty{-78}{\degreeCelsius}. The colour quickly disappears and the contents of the tube remain colourless. The hot and cold tubes are taken out of their beakers, placed side by side and allowed to come to room temperature. The tubes have an identical medium red-brown colour when they both are at room temperature.
  \begin{parts}
    \part The gases are involved in the reversible reaction:
    \schemestart
    \chemfig{N_2O_4}(g) \arrow{<=>} \chemfig{2N0_2}(g).
    \schemestop
    What evidence exists that the forward and reverse rates are equal at room temperature?
    \vspace{1in}
    \part Can temperature changes affect an equilibrium reaction? How do you know this?
    \vspace{1in}
    \part What evidence shows that the forward and reverse reaction rates are equal at \qty{100}{\degreeCelsius}? If the temperature were raised above \qty{100}{\degreeCelsius}, what would you expect to happen to the colour?
    \vspace{1in}
    \part The balanced equation in part (a) should also include "energy". Consider what happened to the
    colour when a tube was heated. Is the reaction exothermic or endothermic, as written? Explain.
    \vspace{1in}
    \part What gas was predominantly present at low temperatures? What gas was predominantly present
    at high temperatures? How would you describe the chemical composition in a tube when it was at
    room temperature?
    \vspace{1in}
    \part If one tube were filled with pure \chemfig{NO_2}(g) and another tube with pure \chemfig{N_2O_4}(g), what might be true of
    the colours you would expect to see in the tubes after they sit for a minute at the same
    temperature? What evidence do you have that your prediction should occur?
    \vspace{1in}
  \end{parts}
  \question Water is boiling in a kettle at \qty{100}{\degreeCelsius}. Is the system at equilibrium? Explain.
  \vspace{1in}
  \question Some liquid water is present inside a sealed flask at room temperature. Is water evaporating inside the flask? Is the system at equilibrium? Why?
  \vspace{1in}
  \question A chemist wished to prepare pure phosgene, \chemfig{COCl_2}(g), by reacting carbon monoxide, \chemfig{CO}(g), and chlorine gas, \chemfig{Cl_2}(g), according to the reaction
  \schemestart
  \chemfig{CO}(g) + \chemfig{Cl_2}(g) \arrow{<=>} \chemfig{COCl_2}(g).
  \schemestop
  Why will this reaction NOT produce pure \chemfig{COCl_2}(g)? If the chemist could somehow obtain a sample of pure \chemfig{COCl}(g), will it remain pure? Why?
  \newpage

  \question Assume that the simple reaction \schemestart A\arrow{<=>}B \schemestop initially has:
  [A] 1.200 M , [B] = 0.000 M . \chemfig{K_{forward}} = 0.50, \chemfig{K_{reverse}} = 0.10 .
  The following results are produced.
  \begin{center}
    \begin{tabular}{|c| c |c| c| c|}
      \hline
      Time (min) & \chemfig{RATE_{forward}} & \chemfig{RATE_{reverse}} & [A]&  [B] \\
      \hline
      0 & 0.600 & 0.000 & 1.200 & 0.000 \\
      \hline
      1 & 0.300 & 0.060 & 0.600 & 0.600 \\
      \hline
      2 & 0.180 & 0.084 & 0.360 & 0.840 \\
      \hline
      3 & 0.132 & 0.094 & 0.264 & 0.936 \\
      \hline
      4 & 0.113 & 0.097 & 0.226 & 0.974 \\
      \hline
      5 & 0.105 & 0.099 & 0.210 & 0.990 \\
      \hline
      6 & 0.102 & 0.100 & 0.204 & 0.996 \\
      \hline
      7 & 0.101 & 0.100 & 0.202 & 0.998 \\
      \hline
      8 & 0.100 & 0.100 & 0.201 & 0.999 \\
      \hline
      9 & 0.100 & 0.100 & 0.200 & 1.000 \\
      \hline
      10 & 0.100 & 0.100 & 0.200 & 1.000 \\
      \hline
    \end{tabular}
  \end{center}
  \begin{parts}
    \part Plot the values of [A]-versus-time and [B]-versus-time on the same graph.
    \vspace{1in}
    \part When does It appear that equilibrium finally occurs? How did you recognize that equilibrium was attained? What else occurs at equilibrium?
    \vspace{1in}
    \part Is there a time when [REACTANT] = [PRODUCT]? Is [REACTANT] = [PRODUCT] at equilibrium?
    \vspace{1in}
    \part When is the forward rate greatest? What happens to the rate as the [A] decreases?
    \vspace{1in}
    \part What is the numerical value the ratio [PRODUCT] / [REACTANT] at equilibrium?
    \vspace{1in}
  \end{parts}

  \question Now let's see what happens if the previous equilibrium is upset by ADDING an extra 0.6 M of B at the 11-th minute, so as to increase [B] from 1.0 M up to 1.6 M.
  \begin{center}
    \begin{tabular}{|c| c |c| c| c|}
      \hline
      Time (min) & \chemfig{RATE_{forward}} & \chemfig{RATE{reverse}} & [A] & [B] \\
      \hline
      11 & 0.100 & 0.160 & 0.200 & 1.600 \\
      \hline
      12 & 0.130 & 0.154 & 0.260 & 1.540 \\
      \hline
      13 & 0.142 & 0.152 & 0.284 & 1.516 \\
      \hline
      14 & 0.147 & 0.151 & 0.294 & 1.506 \\
      \hline
      15 & 0.149 & 0.150 & 0.297 & 1.503 \\
      \hline
      16 & 0.150 & 0.150 & 0.299 & 1.501 \\
      \hline
      17 & 0.150 & 0.150 & 0.300 & 1.500 \\
      \hline
      18 & 0.150 & 0.150 & 0.300 & 1.500 \\
      \hline
    \end{tabular}
  \end{center}
  \begin{parts}
    \part Extend the graph you plotted in Exercise 6, part (a), to include the above data.
    \vspace{1in}
    \part When is equilibrium re-established?
    \vspace{1in}
    \part What is the numerical value of the ratio [PRODUCT] /  [REACTANT] at equilibrium?
    \vspace{1in}
    \part When equilibrium IS re-established, what has changed from the previous equilibrium? What remains unchanged?
    \vspace{1in}
  \end{parts}
  \question In the reaction \schemestart \chemfig{3 C_2H_2}(g) \arrow{<=>} \chemfig{C_6H_6}(g) \schemestop , will the ratio of [\chemfig{C_2H_2}(g)] to [\chemfig{C_6H_6}(g)] at equilibrium be 3:1?
  \newpage
  \question What things must be true at equilibrium?
  \vspace{1in}
  \question Ozone (\chemfig{0_3}) and oxygen (\chemfig{O_2}) molecules can exist in equilibrium: \schemestart \chemfig{2 O_3}(g) \arrow{<=>} \chemfig{3 O_2}(g)\schemestop . If 2 mol of \chemfig{O_3} react for every 2 mol of \chemfig{O_2} reacting in a container, does equilibrium exist in the container? Why?
  \vspace{1in}
  \question The following equilibrium occurs : \schemestart 2 NOCl(g) \arrow{<=>} 2 NO(g) + \chemfig{Cl_2}(g)\schemestop . A gaseous mixture of \chemfig{NOCl, NO, Cl_2} is put in a container. After a few minutes it is found that two moles of NOCl react for every three moles of products which react. Is the mixture at equilibrium? Why?

\end{questions}

\end{document}
