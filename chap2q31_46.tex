\documentclass[fleqn]{exam}
\usepackage{chemfig}
\usepackage{siunitx}

\begin{document}

\begin{questions}
  \setcounter{question}{30}

  \question Write the equilibrium expressions for the following.
  \begin{parts}
    \part \schemestart \chemfig{2 lCl}(g) \arrow{<=>} \chemfig{l_2}(g) + \chemfig{Cl_2}(g) \schemestop
    \vspace{1in}
    \part \schemestart \chemfig{N_2}(g) + \chemfig{O_2}(g) \arrow{<=>} 2 NO(g) \schemestop
    \vspace{1in}
    \part \schemestart \chemfig{3 O_2}(g) \arrow{<=>} \chemfig{2 O_3}(g) \schemestop
    \vspace{1in}
    \part \schemestart \chemfig{2 Bi^{3+}}(aq) + 3 \chemfig{H_2S}(g) \arrow{<=>} \chemfig{Bi_2S_3}(s) + 6 \chemfig{H^+}(aq) \schemestop
    \vspace{1in}
    \part \schemestart \chemfig{CaCO_3}(s) \arrow{<=>} CaO(s) + \chemfig{CO_2}(g) \schemestop
    \vspace{1in}
    \part \schemestart \chemfig{CaC_2}(s) + \chemfig{2 H_2O}(l) \arrow{<=>} \chemfig{C_2H_2}(g) + \chemfig{Ca{(}OH{)}_2}(s) \schemestop
    \vspace{1in}
    \part \schemestart \chemfig{C_6H_6}(l) + \chemfig{Br_2}(l) \arrow{<=>} \chemfig{C_6H_5Br}(l) + HBr(g) \schemestop
    \vspace{1in}
    \part \schemestart \chemfig{Cu}(s) + \chemfig{2 Ag^2}(aq) \arrow{<=>} \chemfig{Cu^{2+}}(aq) + 2 Ag(s) \schemestop
    \vspace{1in}
    \part \schemestart \chemfig{4 NH_3}(g) + 5 \chemfig{O_2}(g) \arrow{<=>} 6 \chemfig{H_2O}(g) + 4 NO(g) \schemestop
    \vspace{1in}
    \part \schemestart \chemfig{H_2}(g) + 1/2 \chemfig{O_2}(g) \arrow{<=>} \chemfig{H_2O}(l) \schemestop
    \vspace{1in}
  \end{parts}

  \question Write the \chemfig{K_{eq}} expression for:
  \begin{parts}
    \part \schemestart \chemfig{N_2O_4}(g) \arrow{<=>} \chemfig{2 NO_2}(g) \schemestop , and
    \vspace{1in}
    \part \schemestart \chemfig{2 NO_2}(g) \arrow{<=>} \chemfig{N_2O_4}(g) \schemestop .
    \vspace{1in}
  \end{parts}
  Examine the relationship between the \chemfig{K_{eq}} expressions for equations (a) and (b) of this question.
  If \chemfig{K_{eq}} = 10.0 for equation (a), what would be the value of \chemfig{K_{eq}} for equation (b)?

  \newpage

  \question Write the \chemfig{K_{eq}} expression for:
  \begin{parts}
    \part \schemestart \chemfig{SO_2}(g) + 1/2 \chemfig{O_2}(g) \arrow{<=>} \chemfig{SO_3}(g) \schemestop, and
    \vspace{1in}
    \part \schemestart \chemfig{2 SO_2}(g) + \chemfig{O_2}(g) \arrow{<=>} \chemfig{2 SO_3}(g) \schemestop .
    \vspace{1in}
  \end{parts}

  Examine the which exists between the \chemfig{K_{eq}} expressions for equations (a) and (b) of this question.
  If \chemfig{K_{eq}} = 3 for equation (a), what would be the value of \chemfig{K_{eq}} for equation (b)?
  \vspace{1in}

  \question Which way will the equilibrium \schemestart \chemfig{CaCO_3}(g) + \chemfig{CO_2}(g) + \chemfig{H_2O}(l) \arrow{<=>} \chemfig{Ca^{2+}}(aq) + 2 \chemfig{HCO_3^{-}}(aq) + 40 kJ \schemestop \newline shift if
  \begin{parts}
    \part more \chemfig{CO_2}(g) is added?
    \vspace{1in}
    \part more \chemfig{CaCO_3}(s) is added?
    \vspace{1in}
    \part \chemfig{Ca^{2+}}(aq) is removed?
    \vspace{1in}
    \part heat is added?
    \vspace{1in}
  \end{parts}

  \question Rearrange the following equations to solve in terms of the concentrations indicated in bold.

  \begin{parts}

    \part \chemfig{K_{eq}} = \boldmath  [\chemfig{H_3O^+}] \unboldmath [\chemfig{F^{-}}] / [HF]
    \vspace{1in}
    \part \chemfig{K_{eq}} = [\chemfig{H_3O^+}][\chemfig{F^{-}}] /  \boldmath [\chemfig{HF}]  \unboldmath
    \vspace{1in}
    \part \chemfig{K_{eq}} = \boldmath [\chemfig{NO_2}]\chemfig{^2} \unboldmath / [\chemfig{NO}]\chemfig{^2}[\chemfig{O_2}]
    \vspace{1in}
    \part \chemfig{K_{eq}} = [\chemfig{NO_2}]\chemfig{^2}  / \boldmath [\chemfig{NO}]\chemfig{^2} \unboldmath [\chemfig{O_2}]
    \vspace{1in}
    \part \chemfig{K_{eq}} = [\chemfig{NH_3}]\chemfig{^2} / [\chemfig{N_2}] \boldmath [\chemfig{H_2}]\chemfig{^3} \unboldmath
    \vspace{1in}
    \part \chemfig{K_{eq}} = [\chemfig{N_2O_4}] / \boldmath [\chemfig{NO_2}]\chemfig{^2} \unboldmath
    \vspace{1in}
    \part \chemfig{K_{eq}} = \boldmath [\chemfig{NH_3}]\chemfig{^2} \unboldmath / [\chemfig{N_2}][\chemfig{H_2}]\chemfig{^3}
    \vspace{1in}
    \part \chemfig{K_{eq}} = [\chemfig{PCl_3}]\chemfig{^4} / [\chemfig{P_4}] \boldmath [\chemfig{Cl_2}]\chemfig{^6} \unboldmath
    \vspace{1in}
  \end{parts}

  \question Consider the following equilibria. \newline
  i) \schemestart \chemfig{2 NO_2}(g) \arrow{<=>} \chemfig{N_2O_4}(g) 0 . \chemfig{K_{eq}} = 2.2 \schemestop \newline
  ii) \schemestart \chemfig{Cu^{2+}}(aq) + 2 Ag(s) \arrow{<=>} Cu(s) + 2 \chemfig{Ag^+}(aq) ; \chemfig{K_{eq}} = \chemfig{1 x 10^{-15}} \schemestop \newline
  ii) \schemestart \chemfig{Pb^{2+}}(aq) + 2 \chemfig{Cl^{-}}(aq) \arrow{<=>} \chemfig{PbCl_2}(s) \chemfig{K_{eq}} = \chemfig{6.3 x 10^4} \schemestop \newline
  iv) \schemestart \chemfig{SO_2}(g) + 1/2 \chemfig{O_2}(g) \arrow{<=>} \chemfig{SO_3}(g) ; \chemfig{K_{eq}} = 110 \schemestop \newline

  \begin{parts}
    \part Which equilibrium favours products to the greatest extent?
    \vspace{1in}
    \part Which equilibrium favours reactants to the greatest extent?
    \vspace{1in}
  \end{parts}

  \question In the reaction \schemestart A + B \arrow{<=>} C + D + 100 kJ \schemestop , what happens to the value of \chemfig{K_{eq}} if the temperature is INCREASED?
  \vspace{1in}
  \question If the value of \chemfig{K_{eq}} DECREASES when the temperature DECREASES, is the reaction EXOTHERMIC or ENDOTHERMIC?
  \vspace{1in}
  \question In the reaction \schemestart P + Q + 150 kJ \arrow{<=>} R + S \schemestop , what happens to the value of \chemfig{K_{eq}} if the temperature is DECREASED?
  \vspace{1in}
  \question In the reaction \schemestart W + X + 100 kJ \arrow{<=>} Y + Z \schemestop , what happens to the Value of \chemfig{K_{eq}} if the [X] is INCREASED?
  \vspace{1in}
  \question If the value of \chemfig{K_{eq}} INCREASES when the temperature DECREASES, is the reaction EXOTHERMIC or ENDOTHERMIC?
  \vspace{1in}
  \question In Exercises 21-23, describe the effect on Kea of the changes indicated. Write INC for increase, DEC
  for decrease and NC for no change.
  \vspace{3in}
  \question In Exercise 23, assume that the bold species Sn(s) instead of \chemfig{CO_2}(g). Now redo the Exercise, describing the effect on the species in bold and the value of \chemfig{K_{eq}} when the changes indicated occur.
  \vspace{1in}
  \question In the equilibrium \schemestart \chemfig{KCl}(s) + 17 kJ \arrow{<=>} \chemfig{K^+}(aq) + \chemfig{Cl^{-}} (aq)\schemestop, which way will the equilibrium shift and what is the effect on the value of \chemfig{K_{eq}} when
  (i) more \chemfig{K^+}(aq) is added?
  (i) the temperature is decreased?
  (ii) more KCl(s) is added?
  \vspace{1in}
  \question An equilibrium \schemestart \chemfig{A}(aq) + 2 B(q) \arrow{<=>} 2 C(aq) + 2 D(aq)\schemestop has
  \chemfig{K_{eq}} = 0.25 at 100°C and \chemfig{K_{eq}} = 0.15 at 200°C
  State whether the forward reaction is endothermic or exothermic and explain why.
  \vspace{1in}
  \question Examine the following graphs for the equilibrium \schemestart \chemfig{3O_2} \arrow{<=>}  \chemfig{2 O_3} \schemestop .

  Is the equilibrium endothermic or exothermic, as written ? Explain.

\end{questions}

\end{document}
