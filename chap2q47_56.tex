\documentclass[fleqn]{exam}
\usepackage{chemfig}
\usepackage{siunitx}

\begin{document}

\begin{questions}
  \setcounter{question}{46}

  \question A 1.0 L reaction vessel contains 0.750 mol of CO(g) and 0.275 mol of \chemfig{H_2O}(g). After 1 h, equilibrium is
  reached according to the equation \schemestart \chemfig{CO}(g) + \chemfig{H_2O}(g) \arrow{<=>} \chemfig{CO_2}(g) + \chemfig{H_2}\schemestop(g). Analysis shows 0.250 mol of \chemfig{CO_2} present at equilibrium. What is \chemfig{K_{eq}} for the reaction?
  \vspace{1in}
  \question A 5.0 L reaction vessel was initially filled with 6.0 mol of \chemfig{SO_2}, 2.5 mol of \chemfig{NO_2} and 1.0 mol of \chemfig{SO_3}. After equilibrium was established according to the equation \schemestart \chemfig{SO_2}(g) + \chemfig{NO_2}(g) \arrow{<=>} \chemfig{SO_3}(g) + NO\schemestop(g), the vessel was found to contain 3.0 mol of \chemfig{SO_3}. What is Key for the reaction?
  \vspace{1in}
  \question Consider the equilibrium \schemestart \chemfig{N_2}(g) + 3 \chemfig{H_2}(g) \arrow{<=>} 2 \chemfig{NH_3}\schemestop(g).
  \begin{parts}
    \part At a certain temperature 3.0 mol of \chemfig{N_2} and 2.0 mol of \chemfig{H_2} are put into a 5.0 L container. At equilibrium the concentration of \chemfig{NH_3} is 0.020 M. Calculate \chemfig{K_{eq}} for the reaction.
    \vspace{1in}
    \part At a different temperature, 6.0 mol of \chemfig{NH_3} were introduced into a 10.0 L container. At equilibrium 2.0 mol of \chemfig{NH_3} were left. Calculate \chemfig{K_{eq}} for the reaction.
    \vspace{1in}
  \end{parts}

  \setcounter{question}{55}

  \question A student obtained the following data at 25°C while studying the equilibrium \newline
  \schemestart \chemfig{2 TI^+}(aq) + Cd(s) \arrow{<=>} \chemfig{2 TI}(s) + \chemfig{Cd^{2+}}\schemestop(aq).

    \begin{center}
      \begin{tabular}{|c|c|c|}
        \hline
        Volume Moles & \chemfig{TI^+} Moles & \chemfig{Cd^{2+}} \\
        \hline
        \hline
        1.00 L &  0.316 & 0.414 \\
        \hline
        5.00 L & ? & 0.339 \\
        \hline
      \end{tabular}
    \end{center}
    Calculate the number of moles of \chemfig{TI^+} present in the second data set.

  \end{questions}

\end{document}
