\documentclass[fleqn]{exam}
\usepackage{chemfig}
\usepackage{siunitx}

\begin{document}

\begin{questions}
  \setcounter{question}{49}

  \question At a certain temperature, \chemfig{K_{eq}} = 4 for the reaction \schemestart \chemfig{2 HF}(g) \arrow{<=>} \chemfig{H_2}(g) + \chemfig{F_2}\schemestop(g). Predict the direction in which the equilibrium will shift, if any, when the following systems are introduced into a 5.0 L bulb.
  \begin{parts}
    \part 3.0 mol of HF, 2.0 mol of \chemfig{H_2} and 4.0 mol of \chemfig{F_2}
    \vspace{1.5in}
    \part 0.20 mol of HF, 0.50 mol of \chemfig{H_2} and 0.60 mol of \chemfig{F_2}
    \vspace{1.5in}
    \part 0.30 mol of HF, 1.8 mol of \chemfig{H_2} and 0.20 mol of \chemfig{F_2}
    \vspace{1.5in}
  \end{parts}
  \newpage

  \question At a Certain temperature, \chemfig{K_{eq}} = 75 for the reaction \schemestart \chemfig{2 O_3}(g) \arrow{<=>} \chemfig{3 O_2}\schemestop(g). Predict the direction in which the equilibrium will shift, if any, when the following systems are introduced into a 10.0 L bulb.
  \begin{parts}
    \part 0.60 mol of \chemfig{O_3} and 3.0 mol of \chemfig{O_2}
    \vspace{1.5in}
    \part 0.050 mol of \chemfig{O_3} and 7.0 mol of \chemfig{O_2}
    \vspace{1.5in}
    \part 1.5 mol of \chemfig{O_3} and no \chemfig{O_2}
    \vspace{1.5in}
  \end{parts}

  \question \chemfig{K_{eq}} = 5.0 at a certain temperature for the reaction \schemestart \chemfig{2 SO_2}(g) + \chemfig{O_2}(g) \arrow{<=>} \chemfig{2 SO_3}\schemestop(g). A certain amount of \chemfig{SO_3}(g) was placed in a 2.0 L reaction vessel. At equilibrium the vessel contained 0.30 mol of \chemfig{O_2}(g). What concentration of \chemfig{SO_3}(g) was originally placed in the vessel?
  \vspace{1.5in}
  \newpage

  \question \chemfig{K_{eq}} = 35.0 for the reaction  \schemestart \chemfig{PCl_5}(g) \arrow{<=>} \chemfig{PCl_3}(g) + \chemfig{Cl_2}\schemestop(g). If you have [\chemfig{PCl_5}] = \num{1.34d-3} M and
  [\chemfig{PCl_3}] = 0.205 M at equilibrium in a certain vessel, what is the equilibrium concentration of \chemfig{Cl_2}(g)?
  \vspace{1.5in}

  \question \chemfig{K_{eq}} = 125 for  \schemestart \chemfig{H_2}(g) + \chemfig{l_2}(g) \arrow{<=>} 2 Hl\schemestop(g) at a certain temperature. If 0.15 mol of Hl, 0.034 mol of \chemfig{H_2} and 0.096 mol of \chemfig{l_2} are introduced into a 10 L vessel, will the reaction proceed to the reactant side or product side as the reaction attempts to reach equilibrium?
  \vspace{1.5in}

  \question A reaction mixture at equilibrium,  \schemestart CO(g) + \chemfig{H_2O}(g) \arrow{<=>} \chemfig{CO_2}(g) + \chemfig{H_2}\schemestop(g), contains 1.00 mol of \chemfig{H_2}, 2.00 mol of \chemfig{CO_2}, 2.00 mol of CO and 2.00 mol of \chemfig{H_2O} in a 2.00 L bulb. If 1.00 mol of \chemfig{H_2} is added to the system, calculate the [CO] which will exist when equilibrium is regained.
  \vspace{1.5in}
  \setcounter{question}{56}


  \question When 0.50 mol of NOCl(g) was put into a 1.0 L flask and allowed to come to equilibrium, 0.10 mol of \chemfig{Cl_2}(g) was found. What is \chemfig{K_{eq}} for the reaction  \schemestart \chemfig{2 NOCl}(g) \arrow{<=>} \chemfig{2 NO}(g) + \chemfig{Cl_2}\schemestop(g)?
  \vspace{1.5in}
  \newpage

  \question \chemfig{K_{eq}} = 7.5 for  \schemestart \chemfig{2 H_2}(g) + \chemfig{S_2}(g) \arrow{<=>} \chemfig{2 H_2S}\schemestop(g). A certain amount of \chemfig{H_2} was added to a 2.0 L flask and allowed to come to equilibrium. At equilibrium, 0.072 mol of \chemfig{H_2} was found. How many moles of \chemfig{H_2S} were originally added to the flask?
  \vspace{1.5in}

  \question A reaction mixture at equilibrium,  \schemestart \chemfig{CO_2}(g) + \chemfig{H_2}(g) \arrow{<=>} CO(g) + \chemfig{H_2O}\schemestop(g), contained 4.00 mol of \chemfig{CO_2}, 1.50 mol of \chemfig{H_2}, 3.00 mol of CO and 2.50 mol of \chemfig{H_2O} in a 5.0 L container. How many moles of \chemfig{CO_2} would have to be removed from the system in order to reduce the amount of CO to 2.50 mol?
  \vspace{1.5in}

  \question \chemfig{K_{eq}} = 49.5 for  \schemestart \chemfig{H_2}(g) + \chemfig{l_2}(g) \arrow{<=>} 2 Hl\schemestop(g) at a certain temperature. If 0.250 mol of \chemfig{H_2}(g) and 0.250 mol of \chemfig{l_2}(g) are placed in a 10.0 L vessel and permitted to react, what will be the concentration of each substance at equilibrium?
  \vspace{1.5in}
  \newpage

  \question The equilibrium constant for the reaction  \schemestart \chemfig{N_2}(g) + \chemfig{3 H_2}(g) \arrow{<=>} \chemfig{2 NH_3}\schemestop(g) is 3.0 at a certain temperature. Enough \chemfig{NH_3}(g) was added to a 5.0 L container such that at equilibrium the container was found to contain 2.5 mol of \chemfig{N_2}(g). How many moles of \chemfig{NH_3}(g) were put into the container?
  \vspace{1.5in}

  \question \chemfig{K_{eq}} = 1.00 for  \schemestart \chemfig{N_2O_2}(g) + \chemfig{H_2}(g) \arrow{<=>} \chemfig{N_2O}(g) + \chemfig{H_2O}\schemestop(g). If 0.150 mol of \chemfig{N_2O}(g) and 0.250 mol of \chemfig{H_2O}(g) were introduced into a 1.00 L bulb and allowed to come to equilibrium, what concentration \chemfig{N_2O_2}(g) was present at equilibrium?
  \vspace{1.5in}

  \question A reaction mixture at equilibrium, \schemestart \chemfig{H_2}(g) + \chemfig{l_2}(g) \arrow{<=>} 2 Hl\schemestop(g), contains 0.150 mol of \chemfig{H_2}(g), 0.150 mol of \chemfig{l_2}(g) and 0.870 mol of Hl(g) in a 10.0 L vessel. If 0.400 mol of Hl(g) is added to this system and the system is allowed to come to equilibrium again, what will be the new concentrations of \chemfig{H_2}, \chemfig{l_2} and Hl?
  \vspace{1.5in}

  \question A reaction mixture, \schemestart 2 NO(g) + \chemfig{O_2}(g) \arrow{<=>} \chemfig{2 NO_2}\schemestop(g), contained 0.240 mol of NO(g), 0.0860 mol of \chemfig{O_2}(g) and 1.20 mol of \chemfig{NO_2}(g) when at equilibrium in a 2.00 L bulb. How many moles of \chemfig{O_2}(g) had to be added to the mixture to increase the number of moles of \chemfig{NO_2}(g) to 1.28 when equilibrium was re-established?
  \vspace{1.5in}

  \question A reaction mixture, \schemestart 2 lCl(g) + \chemfig{H_2}(g) \arrow{<=>} \chemfig{l_2}(g) + 2 HCl\schemestop(g), was found to contain 0.500 mol of lC(g). 0.0560 mol of \chemfig{H_2}(g), 1.360 mol of \chemfig{l_2}(g) and 0.800 mol of HCl(g) at equilibrium in a 1.00 L bulb. How many moles of lCl(g) would have to be removed in order to reduce the [HCl(g)] to 0.680 M when equilibrium is re-established?
  \vspace{1.5in}

  \question (Nasty!) \chemfig{K_{eq}} = 100 at a certain temperature for \schemestart \chemfig{CH_4}(g) + 2 \chemfig{H_2S}(g) \arrow{<=>} \chemfig{CS_2}(g) + 4 \chemfig{H_2}\schemestop(g). Some \chemfig{CH_4} and \chemfig{H_2S} were introduced into a 1.0 L bulb and at equilibrium 0.10 mol of \chemfig{CH_4} and 0.30 mol of \chemfig{H_2S} were found. What was [\chemfig{CS_2}] at equilibrium?

\end{questions}

\end{document}
