\documentclass[fleqn]{exam}
\usepackage{chemfig}
\usepackage{siunitx}

\begin{document}

\begin{questions}

  \setcounter{question}{25}


  \question When adding a salt to precipitate a cation from a mixture of ions, why must the salt be soluble?
  \vspace{1.5in}
  \question What ions could be present in solution it separate samples of It gave a precipitate when: a
  \begin{parts}
    \part either \chemfig{SO^{2-}_4} or \chemfig{OH^{-}} is added?
    \vspace{1.5in}
    \part \chemfig{SO^{2-}_4}, is added, but none when \chemfig{OH^{-}} is added?
    \vspace{1.5in}
  \end{parts}

  \question A solution contains only one of \chemfig{Ag^{+}} or \chemfig{Pb^{2+}}. Is it possible to use a precipitation procedure based on your Solubility Table to determine which ion is present? If so, how? If not, why?
  \vspace{1.5in}

  \question A solution contains \chemfig{Al^{3+}} and \chemfig{Ag^+}. What compounds could be added, and in what order, to separate these ions? You must specify the complete compound which will be added, not just the anion contained in the compound. Do not write a complete experimental procedure.
  \vspace{1.5in}

  \question A solution contains \chemfig{Sr^{2+}}, \chemfig{Ca^{2+}} and \chemfig{Ag^+}. What compounds could be added, and in what order, to separate these ions?
  \vspace{1.5in}

  \question A solution contains \chemfig{Mg^{2+}}, \chemfig{Pb^{2+}} and \chemfig{Zn^{2+}}. What compounds could be added, and in what order, to separate these ions?
  \vspace{1.5in}

  \question A solution contains \chemfig{Fe^{3+}}, \chemfig{Ca^{2+}}, \chemfig{Ag^+} and \chemfig{Be^{2+}}. What compounds could be added, and in what order, to separate these ions?
  \vspace{1.5in}

  \question Using your results from Exercise 29, write an experimental procedure for analyzing a solution which can only contain \chemfig{Ag^+} and \chemfig{Al^{3+}}, but might contain one, both or neither of these ions.
  \vspace{1.5in}

  \question Using your results from Exercise 30, write an experimental procedure for analyzing a solution which can only contain \chemfig{Sr^{2+}}, \chemfig{Ca^{2+}} and \chemfig{Ag^+}, but might contain any number of these ions.
  \vspace{1.5in}

  \question You are asked to identify the ions present in a particular solution. The ions which may be present are: \chemfig{l^{-}}, \chemfig{SO^{2-}_4}, and \chemfig{OH^{-}}.
  \begin{parts}
    \part What is the name given to the process of identifying the chemical substances in a sample?
    \vspace{1.5in}
    \part You are to perform the identification using only the following reagents (that is, test chemicals): \chemfig{AgNO_3}, \chemfig{Ca{(}NO_3{)}_2} and \chemfig{Mg{(}NO_3{)}_2}. Which reagent must be added first? Explain why.
    \vspace{1.5in}
    \part How would you complete the analysis of the solution?
    \vspace{1.5in}
  \end{parts}

  \question A solution is known to contain one or more of the ions: \chemfig{S^{2-}}, \chemfig{OH^{-}} , \chemfig{Cl^{-}} and \chemfig{CO^{2-}_3}. You are to identify the ions present using only the reagents: \chemfig{AgNO_3} \chemfig{Ba{(}NO_3{)}_2} , \chemfig{Cu{(}NO_3{)}_2} and \chemfig{Sr{(}NO_3{)}_2}. Briefly describe a procedure which could be used to analyze the anions in the solution.
  \vspace{1.5in}
  \newpage

  \question You have a sample of a solution which contains \chemfig{Ba^{2+}}. You add \chemfig{Na_2SO_4}, to the sample until no more \chemfig{BaSO_4}(s) will precipitate, then filter, dry and weigh the precipitate. Your results are as follows. \newline
  volume of sample = 25.0 mL \newline
  mass of filter paper = 1.21 g \newline
  mass of filter paper + \chemfig{BaSO_4} (dry) = 3.75 g \newline

  Calculate:
  \begin{parts}
    \part the moles of \chemfig{BaSO_4} produced.
    \vspace{1.5in}
    \part the concentration of \chemfig{Ba^{2+}} in the original solution.
    \vspace{1.5in}
  \end{parts}

  \question A solution contains an unknown amount of \chemfig{Pb^{2+}}. If 4.28 g of \chemfig{PbSO_4}(s) are obtained from 100.0 mL of the solution, what is the [\chemfig{Pb^{2+}}] in the solution?
  \vspace{1.5in}

  \question Addition of phosphate ions to a 25.00 mL sample of a solution containing \chemfig{Ag^+} ions produces 1.57 g of \chemfig{Ag_3PO_4}(s). What is the [\chemfig{Ag^+}] in the original solution?

\end{questions}

\end{document}
