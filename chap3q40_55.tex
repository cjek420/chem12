\documentclass[fleqn]{exam}
\usepackage{chemfig}
\usepackage{siunitx}

\begin{document}

\begin{questions}
  \setcounter{question}{39}
  \question Write the EQUILIBRIUM EQUATION for a saturated solution of the following salts AND the corresponding SOLUBILITY PRODUCT EXPRESSIONS.
  \begin{parts}
    \part \chemfig{BaSO_4}(s)
    \vspace{1in}
    \part \chemfig{MgF_2}(s)
    \vspace{1in}
    \part \chemfig{Ag_2S}(s)
    \vspace{1in}
    \part \chemfig{Cu{(}lO_3{)}_2}(s)
    \vspace{1in}
  \end{parts}

  \question Which of the following salts is the most soluble? Which is the least?
  \newline
  AgCl ; \chemfig{K_{sp}} = \num{1.8d-10} \newline
  Aal ; \chemfig{K_{sp}} = \num{8.5d-17} \newline
  AgBr; \chemfig{K_{sp}} = \num{5.4d-13}
  \vspace{1in}

  \question A solution in equilibrium with a precipitate of \chemfig{FeCO_3} contains \num{5d-6} M \chemfig{Fe^{2+}} and \num{6d-6} M \chemfig{CO^{2-}_3}. Calculate \chemfig{K_{sp}} for \chemfig{FeCO_3}.
  \vspace{1in}

  \question What is the concentration of \chemfig{Zn^{2+}} ions in a saturated solution made by shaking ZnS(s) with water?
  \vspace{1in}

  \question How many grams of \chemfig{PbSO_4}(s) Will dissolve in 5.0 L of water?
  \vspace{1in}

  \setcounter{question}{46}

  \question Calculate the molar solubility of \chemfig{Ag_2CrO_4}.
  \vspace{1in}

  \setcounter{question}{48}

  \question Calculate the solubility of \chemfig{Fe{(OH)}_2} in grams per litre.
  \vspace{.85in}

  \question A solution in equilibrium with a precipitate of \chemfig{Ag_2S} contained \num{1.6d-16} M \chemfig{S^{2-}} and \num{2.6d-17} M \chemfig{Ag^+}.
  Calculate the solubility product of \chemfig{Ag_2S}.
  \vspace{.85in}

  \question A small piece of the mineral smithsonite, \chemfig{ZnCO_3}, with a mass of 0.000 14 g just barely dissolves in
  100.0 mL of water. Calculate \chemfig{K_{sp}}
  for \chemfig{ZnCO_3}.
  \vspace{.85in}

  \question What is the concentration of \chemfig{OH^{-}} in a saturated solution of \chemfig{Zn{(OH)}_2}? \chemfig{K_{sp}} = \num{4.1d-17} for \chemfig{Zn{(OH)}_2}.
  \vspace{.85in}

  \setcounter{question}{54}

  \question The data below was obtained when a student combined various solutions of \chemfig{Mn{(NO_3)}_2} and KOH.
  \begin{center}
    \begin{tabular}{|c| c |c|}
      \hline
      Trial & [\chemfig{Mn^{2+}}] & [\chemfig{OH^{-}}] \\ [1.0ex]
      \hline
      1 &  \num{2.1d-5}M & \num{1.0d-4}M \\
      \hline
      2 & \num{7.8d-4}M & ? \\
      \hline
    \end{tabular}
  \end{center}

  What is the value of the [\chemfig{OH^{-}}] in Trial 2?


\end{questions}

\end{document}
