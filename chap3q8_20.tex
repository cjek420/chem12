\documentclass[fleqn]{exam}
\usepackage{chemfig}
\usepackage{siunitx}

\begin{document}

\begin{questions}
  \setcounter{question}{7}

  \question Aluminum fluoride, \chemfig{AlF_3} has a solubility of 5.59 g/L of solution at 20°C. Express this solubility in moles per litre.
  \vspace{1.5in}
  \question Lead (II) chloride, \chemfig{PbCl_2}, has solubility of 0.99 g/100.0 mL of solution at 20°C. Calculate the molar solubility of \chemfig{PbCl_2}.
  \vspace{1.5in}
  \question The molar solubility Of \chemfig{MgCO_3} is \num{1.26d-3} M at 25°C. Express this value in grams per litre.
  \vspace{1.5in}
  \question The molar solubility of \chemfig{Ag_2CO_3} is \num{1.2d-4} M at 25°C. Express this value in grams per 100.0 mL.
  \vspace{1.5in}
  \question Chromium (VI) oxide, CrO3(s), has a solubility of 92.6 g in 150.0 mL of solution at 0°c. Calculate the molar solubility of \chemfig{CrO_3}.
  \vspace{1.5in}
  \question Silver chlorite, \chemfig{AgClO_2}, has a molar solubility of 0.014 M at 25°c. What mass of \chemfig{AgClO_2} is contained in 50. 0 mL of saturated \chemfig{AgClO_2}?
  \vspace{1.5in}
  \question Manganese (II) chloride, \chemfig{MnCl_2}, has a molar solubility of 5.75 M at 0°C. If 125 mL of saturated \chemfig{MnCl_2} is evaporated to dryness, what mass of \chemfig{MnCl_2} will be left?
  \newpage
  \question A chemistry student was assigned the task of determining the solubility of potassium chloride, KCl. She added an excess of solid KCl to water, stirred, and let the solution sit overnight. The next day, she pipetted a 25.00 mL portion of the saturated solution into a pre-weighed evaporating dish, determined the combined and re determined the mass, carefully boiled off the water present, allowed the residue to cool mass of the evaporating dish and residue. The data obtained is given below.

  \begin{center}
    temperature of solution = 22.5°C \newline
    mass of evaporating dish = 54.87 g \newline
    mass of solution and evaporating dish = 84.84 g \newline
    mass of residue and evaporating dish = 62.59 g \newline
  \end{center}

  Calculate:
  \begin{parts}
    \part the mass of 25.00 mL of the solution.
    \vspace{1.25in}
    \part the mass of KCl in 25.00 mL of solution.
    \vspace{1.25in}
    \part the mass of water in 25.00 mL of solution.
    \vspace{1.25in}
    \part the mass of KCl which can dissolve in 100.0 g of water at 22.5°C.
    \vspace{1.25in}
    \part the molar solubility of KCl, expressed in moles of KCl per litre of solution.
  \end{parts}
  \newpage

  \question The following data was obtained when a saturated solution of aqueous ammonium sulphate,
  \chemfig{{(}NH_4{)}_2SO_4}(aq), was poured into a beaker and evaporated to dryness.
  \begin{center}
    temperature of solution = 25°C \newline
    volume of solution used 70.0 mL \newline
    mass of beaker = 87.23 g \newline
    mass of original solution and beaker = 147.42 g \newline
    mass of beaker and dried (NH4)2S04 = 104.08 g \newline
  \end{center}
  Calculate:
  \begin{parts}
    \part The mass of the solution.
    \vspace{1.25in}
    \part The mass of ammonium sulphate in the solution.
    \vspace{1.25in}
    \part The mass of water in the solution.
    \vspace{1.25in}
    \part The mass of ammonium sulphate which could be dissolved in 100.0 g of water.
    \vspace{1.25in}
    \part The molar concentration of the ammonium sulphate solution.
  \end{parts}

  \newpage
  \question Examine the following diagram:
  \begin{parts}
    \part Which salt is the most soluble at 60°C?
    \vspace{1.25in}
    \part If you put 40 g of KCl into 100g of water at 90°C, will you be able to form a saturated solution?
    Explain your answer.
    \vspace{1.25in}
    \part If you heat a saturated solution of calcium acetate, \chemfig{Ca{(}CH_3COO{)}_2}, from 20°C to 80°C, what will you observe?
    \vspace{1.25in}
    \part If you put 20 g of \chemfig{MgCl_2} into 100 g of water at 20°C and gradually heat the solution, what will you observe?
    \vspace{1.25in}
    \part If you dissolve 90 g of both KBr and LiCl in 100 g of water at 90°C and then cool the mixture to 10°C, which salt will form crystals first?
    \vspace{1.25in}
    \part A solution contains 20 g of KCl and 20 g of KBr in 100 g of water at 20°C. If the solution is left open to the air, which salt will form crystals first as the water evaporates?
    \vspace{1.25in}
    \part Make a general statement regarding the change in solubility of LiCl(s) with a change in temperature. What does this imply about shifting the equilibrium: \newline
    \schemestart LiCl(s) \arrow{<=>} \chemfig{Li^+} (aq) + \chemfig{Cl^{-}} \schemestop (aq) \newline
    when the temperature is increased? Is the dissolving of LiCl(s) an endothermic or exothermic process?
    \vspace{1.25in}
    \part Is the dissolving of \chemfig{Ca{(}CH_3COO{)}_2}(s) endothermic or exothermic?
    \vspace{1.25in}
  \end{parts}

  \question Calculate the concentration of all the ions in each of the following solutions.
  \begin{parts}
    \part 0.25 M \chemfig{FeCl_3}
    \vspace{1.5in}
    \part \num{1.5d-3} M \chemfig{Al_2{(}SO_4{)}_3}
    \vspace{1.5in}
    \part 12.0 g of \chemfig{{(}NH_4{)}_2CO_3} in 2.50 L
    \vspace{1.5in}
    \part 0.41 g of \chemfig{Ca{(}OH{)}_2} in 500 mL of aqueous solution
    \vspace{1.5in}
    \part 2.50 g of KBr in 150 mL of aqueous solution
    \vspace{1.5in}
  \end{parts}

  \question
  \begin{parts}
    \part Write an equation showing the equilibrium in a saturated solution of lead (II) bromide, \chemfig{PbBr_2.}
    \vspace{1.5in}
    \part The solubility of \chemfig{PbBr_2} is 0.844 g/100 mL. What is its molar solubility?
    \vspace{1.5in}
    \part Calculate the concentrations of \chemfig{Pb^{2+}}(aq) and \chemfig{Br^{-}} (aq) in a saturated solution of \chemfig{PbBr_2}.
  \end{parts}

  \newpage
  \question Calculate the concentration of all the ions present when
  \begin{parts}
    \part 25.0 mL of water is added to 20.0 mL of 0.35 M \chemfig{Fe^{3+}}.
    \vspace{1.5in}
    \part 50.0 mL of 0.25 M \chemfig{Ag^+} is mixed with 100.0 mL of 0. 10 M \chemfig{NO^{-}_3}.
    \vspace{1.5in}
    \part 15.0 mL of \num{6.5d-5} M \chemfig{Cu^{2+}} is mixed with 40.0 mL of \num{3.2d-3} M \chemfig{Cl^{-}}.
    \vspace{1.5in}
    \part 55.0 mL of 0.185 M \chemfig{MgCl_2} is mixed with 25.0 mL of \num{4.8d-2} M \chemfig{CaBr_2}.
    \vspace{1.5in}
    \part 95.0 mL of \num{8.65d-4} M \chemfig{Al{(}NO_3{)}_3} mixed with 15.0 mL of \num{7.50d-6} M \chemfig{Ag_2SO_4}.
    \vspace{1.5in}
    \part 50.0 mL of 0.200 M \chemfig{CaCl_2} is mixed with 50.0 mL of 0.200 M NaCl.
    \vspace{1.5in}
    \part 25.0 mL of 0.360 M \chemfig{NH_4Br} is mixed with 75.0 mL of 0.160 M \chemfig{{(}NH_4{)}_2SO_4}.
    \vspace{1.5in}
    \part 10.0 mL of 0. 100 M \chemfig{Ba{(}NO_3{)}_2} is mixed with 40.0 mL of 0.300 M \chemfig{AgNO_3}.
  \end{parts}

\end{questions}

\end{document}
