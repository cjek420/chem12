\documentclass[fleqn]{exam}
\usepackage{chemfig}
\usepackage{siunitx}

\begin{document}

\begin{questions}
  \setcounter{question}{18}

  \question Balance the following half-reactions.
  \begin{parts}
    \part \schemestart \chemfig{Ce^{4+}} \arrow{<=>} \chemfig{Ce^{2+}} \schemestop
    \vspace{.45in}
    \part \schemestart \chemfig{l_2} \arrow{<=>} \chemfig{l^{-}} \schemestop
    \vspace{.45in}
    \part \schemestart \chemfig{Mn^{2+}} \arrow{<=>} \chemfig{MnO_2} \schemestop \enspace (acidic solution)
    \vspace{.45in}
    \part \schemestart \chemfig{O_2} \arrow{<=>} \chemfig{H_2O_2} \schemestop \enspace (acidic solution)
    \vspace{.45in}
    \part \schemestart \chemfig{S_2O^{2-}_8}  \arrow{<=>} \chemfig{HSO^{-}_4}  \schemestop \enspace (acidic solution)
    \vspace{.45in}
    \part \schemestart \chemfig{H_3AsO_4}  \arrow{<=>} \chemfig{HAsO_2}  \schemestop \enspace (acidic solution)
    \vspace{.45in}
    \part \schemestart \chemfig{H_2SeO_3}  \arrow{<=>} \chemfig{Se}  \schemestop \enspace (acidic solution)
    \vspace{.45in}
    \part \schemestart \chemfig{N_2H_4}  \arrow{<=>} \chemfig{N_2}  \schemestop \enspace (basic solution)
    \vspace{.45in}
    \part \schemestart \chemfig{HO^{-}_2}  \arrow{<=>} \chemfig{O_2}  \schemestop \enspace (basic solution)
    \vspace{.45in}
    \part \schemestart \chemfig{HXeO^{-}_4}  \arrow{<=>} \chemfig{HXeO^{3-}_6}  \schemestop \enspace (basic solution)
    \vspace{.45in}
    \part \schemestart \chemfig{HC_2H_3O_2} \arrow{<=>} \chemfig{C_2H_5OH}  \schemestop \enspace (acidic solution)
    \vspace{.45in}
    \part \schemestart \chemfig{Cr{(}OH{)}_3}  \arrow{<=>} \chemfig{CrO^{2-}_4}  \schemestop \enspace (basic solution)
    \vspace{.45in}
    \part \schemestart \chemfig{CH_3CHO}  \arrow{<=>} \chemfig{CH_2CH_2}  \schemestop \enspace (acidic solution)
  \end{parts}
  \newpage

  \question In the half-reaction \schemestart \chemfig{NO^{-}_2} \arrow{<=>} \chemfig{NO^{-}_3} \schemestop : \newline
  \begin{parts}
    \part calculate the oxidation numbers for N on both sides of the equation.
    \vspace{2in}
    \part calculate ``$\Delta$ON'' (the ``change in Oxidation Number''), and decide on a sign for the value of $\Delta$ON.
    (Hint: the change equals the oxidation number of the nitrogen on the product side minus the oxidation number of the nitrogen on the reactant side.)
    \vspace{2in}
    \part balance the half-reaction in acid solution.
    \vspace{2in}
    \part look at the number of electrons involved and compare this to the value of $\Delta$ON. Is the half-reaction a reduction or oxidation?
  \end{parts}
  \newpage

  \question In the half-reaction  \schemestart \chemfig{MnO^{-}_4} \arrow{<=>} \chemfig{MnO_2} \schemestop : \newline
  \begin{parts}
    \part calculate the oxidation numbers for Mn on both sides of the equation.
    \vspace{2in}
    \part calculate $\Delta$ON and assign a sign for the value of $\Delta$ON.
    \vspace{2in}
    \part balance the half-reaction in acid solution.
    \vspace{2in}
    \part look at the number of electrons involved and compare this to the value of $\Delta$ON. Is the half-reaction a reduction or oxidation?
    \vspace{2in}
  \end{parts}

  \question Summarize the results of the above two exercises by completing the following sentence. \newline
  The OXIDATION NUMBER ? during a REDUCTION reaction and ? during an OXIDATION reaction.
  \vspace{1in}

  \question For each of the half-reactions below \newline
  i) determine the change in oxidation number of the atom in bold type. \newline
  ii) state whether the half-reaction is an oxidation or a reduction.
  \begin{parts}
    \part \schemestart \chemfig{Te} \arrow{<=>} \chemfig{TeO^{-}_4}  \schemestop
    \vspace{1in}
    \part \schemestart \chemfig{ClO^{-}_3}  \arrow{<=>} \chemfig{Cl^{-}}  \schemestop
    \vspace{1in}
    \part \schemestart \chemfig{U^{4+}}  \arrow{<=>} \chemfig{UO^{2+}_2}  \schemestop
    \vspace{1in}
    \part \schemestart \chemfig{C_2H_5OH}  \arrow{<=>} \chemfig{CH_3COOH}  \schemestop  (treat both carbons in CH.COOH identicaly)
    \vspace{1in}
    \part \schemestart \chemfig{PO^{3-}_4}  \arrow{<=>} \chemfig{HPO^{2-}_3}  \schemestop

  \end{parts}

\end{questions}

\end{document}
