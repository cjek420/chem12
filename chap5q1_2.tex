\documentclass[fleqn]{exam}
\usepackage{chemfig}
\usepackage{siunitx}

\begin{document}

\begin{questions}

  \question In the following reactions, indicate the: \newline
  i) species oxidized ii) species reduced iii) oxidizing agent iv) reducing agent.
  \begin{parts}
    \part \schemestart \chemfig{Hg^{2+}} + Mn \arrow{->} Hg + \chemfig{Mn^{2+}} \schemestop
    \vspace{1.75in}
    \part \schemestart \chemfig{H_2} + \chemfig{Sn^{4+}} \arrow{->} \chemfig{2H^+} + \chemfig{Sn^{2+}} \schemestop
    \vspace{1.75in}
    \part \schemestart \chemfig{2 Li} + \chemfig{F_2} \arrow{->} \chemfig{2 Li^+} + \chemfig{2 F^{-}} \schemestop
    \vspace{1.75in}
    \part \schemestart \chemfig{Br_2} + \chemfig{2 Cr^{2+}} \arrow{->} \chemfig{2 Br^{-}} + \chemfig{2 Cr^{3+}} \schemestop
    \vspace{1.75in}
    \part \schemestart \chemfig{2 Fe^{2+}} + \chemfig{Sn^{4+}} \arrow{->} \chemfig{Sn^{2+}} + \chemfig{2 Fe^{3+}} \schemestop
  \end{parts}

  \newpage

  \question When cesium metal is exposed to chlorine gas, a bright flash occurs as the elements react. The product, cesium chloride, is a white solid composed of cesium ions and chloride ions.
  \begin{parts}
    \part Write the balanced overall reaction which occurs between chlorine and cesium.
    \vspace{2in}
    \part Write the half-reactions which occur and identify which half-reaction is the reduction and which is the oxidation.
    \vspace{2in}
    \part Identify the reducing and oxidizing agents.
  \end{parts}

\end{questions}

\end{document}
