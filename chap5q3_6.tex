\documentclass[fleqn]{exam}
\usepackage{chemfig}
\usepackage{siunitx}

\begin{document}

\begin{questions}
  \setcounter{question}{2}

  \question Calculate the oxidation number of the atom in bold type.
  \begin{parts}
    \part \chemfig{H\textbf{N}O_3}
    \vspace{.25in}
    \part \chemfig{\textbf{N}O^{-}_2}
    \vspace{.25in}
    \part \chemfig{\textbf{Cr}O^{2-}_4}
    \vspace{.25in}
    \part \chemfig{\textbf{Cr}_2O^{2-}_7}
    \vspace{.25in}
    \part \chemfig{\textbf{N}H^+_4}
    \vspace{.25in}
    \part \chemfig{\textbf{N}^{-}_3}
    \vspace{.25in}
    \part \chemfig{\textbf{C}_2H_6}
    \vspace{.25in}
    \part \chemfig{\textbf{C}_3H_8}
    \vspace{.25in}
    \part \chemfig{\textbf{Al}{(}OH{)}^{-}_4}
    \vspace{.25in}
    \part \chemfig{\textbf{S}_2F_{10}}
    \vspace{.25in}
    \part \chemfig{\textbf{N}_2O_3}
    \vspace{.25in}
    \part \chemfig{H\textbf{Cl}O_4}
    \vspace{.25in}
    \part \chemfig{H\textbf{Cl}O_3}
    \vspace{.25in}
    \part \chemfig{\textbf{N}_2H^+_5}
    \vspace{.25in}
    \part \chemfig{\textbf{N}H_2OH}
    \vspace{.25in}
    \part \chemfig{\textbf{C}_2O^{2-}_4}
    \vspace{.25in}
    \part \chemfig{K_2\textbf{U}O_4}
    \vspace{.25in}
    \part \chemfig{\textbf{C}_3H_6O}
    \vspace{.25in}
    \part \chemfig{\textbf{S}_8}
    \vspace{.25in}
    \part \chemfig{\textbf{C}_4H_6}
    \vspace{.25in}
  \end{parts}

  \question Assign oxidation numbers to the bold species in each of the following unbalanced reaction equations.
  Then determine which species undergoes oxidation in each reaction.
  \begin{parts}
    \part \schemestart \chemfig{\textbf{Cl}O_2} + \textbf{C} \arrow{->} \chemfig{\textbf{Cl}O^{-}_2} + \chemfig{\textbf{C}O^{2-}_3} \schemestop
    \vspace{1.25in}
    \part \schemestart \chemfig{\textbf{Sn}^{2+}} + \chemfig{Cl^{-}} + \chemfig{\textbf{Br}O^{-}_3} \arrow{->} \chemfig{\textbf{Sn}Cl^{2-}_6} + \chemfig{\textbf{Br}^{-}} \schemestop
    \vspace{1.25in}
    \part \schemestart \chemfig{\textbf{Mn}O^{-}_4} + \chemfig{\textbf{C}_2O^{2-}_4} \arrow{->} \chemfig{\textbf{Mn}O_2} + \chemfig{\textbf{C}O_2} \schemestop
    \vspace{1.25in}
    \part \schemestart \chemfig{\textbf{N}O^{-}_3} + \chemfig{H_2\textbf{Te}} \arrow{->} \textbf{N}O + \chemfig{\textbf{Te}O^{2-}_4} \schemestop
    \vspace{1.25in}
  \end{parts}

  \newpage

  \question Which of the following are redox reactions?
  \begin{parts}

    \part \schemestart \chemfig{l_2} + 5 \chemfig{HOBr} + \chemfig{H_2O} \arrow{->} \chemfig{2 lO^{-}_3} + 5 \chemfig{Br^{-}} + 7 \chemfig{H^+} \schemestop
    \vspace{1.25in}
    \part \schemestart 4 \chemfig{Ag^+} + \chemfig{Cr_2O^{2-}_7} + \chemfig{H_2O} \arrow{->} 2 \chemfig{Ag_2CrO_4} + \chemfig{2H^+} \schemestop
    \vspace{1.25in}
    \part \schemestart \chemfig{KHCO_3} + Hl \arrow{->} Kl + \chemfig{CO_2} + \chemfig{H_2O} \schemestop
    \vspace{1.25in}
    \part \schemestart 2 \chemfig{H_2O} \arrow{->} \chemfig{2H_2} + \chemfig{O_2} \schemestop
    \vspace{1.25in}
    \part \schemestart \chemfig{H_2SO_4} + \chemfig{BaCl_2} \arrow{->} \chemfig{BaSO_4} + 2 HCl \schemestop
    \vspace{1.25in}
    \part \schemestart Fe + \chemfig{H_2SO_4} \arrow{->} \chemfig{FeSO_4} + \chemfig{H_2} \schemestop
    \vspace{1.25in}
  \end{parts}

  \question
  \begin{parts}

    \part Which of \chemfig{Cl_2}, \chemfig{ClO^{-}_4}, \chemfig{Cl^{-}}, \chemfig{ClO^{-}_3}, and \chemfig{Cl_2O} can be produced by reducing \chemfig{ClO^{-}_2}?
    \vspace{1.25in}
    \part Which of \chemfig{NO^{-}_3}, \chemfig{N_2}, \chemfig{NO^{-}_2}, \chemfig{N_2O} and \chemfig{N_2O_3} can be produced by oxidizing NO?

  \end{parts}

\end{questions}

\end{document}
