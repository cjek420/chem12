\documentclass[fleqn]{exam}
\usepackage{chemfig}
\usepackage{siunitx}

\begin{document}

\begin{questions}
  \setcounter{question}{6}

  \question Using the Standard Reduction Potential table, state whether the following species \newline
  can only undergo reduction, or \newline
  can only undergo oxidation, or \newline
  cannot react at all, or \newline
  can undergo either reduction or oxidation (because it is found on both sides of the Table). \newline
  \begin{parts}
    \part \chemfig{Na^+}
    \vspace{.35in}
    \part \chemfig{l^{-}}
    \vspace{.35in}
    \part \chemfig{Cu^+}
    \vspace{.35in}
    \part \chemfig{Sn^{4+}}
    \vspace{.35in}
    \part \chemfig{NO^{-}_3}
    \vspace{.35in}
    \part \chemfig{Hg{(}l{)}}
    \vspace{.35in}
    \part \chemfig{Fe^{2+}}
    \vspace{.35in}
    \part \chemfig{Co^{2+}}
    \vspace{.35in}
    \part \chemfig{Se{(}s{)}}
    \vspace{.35in}
    \part \chemfig{Sn^{2+}}
    \vspace{.35in}
    \part \chemfig{Al{(}s{)}}
    \vspace{.35in}
    \part acidic \chemfig{Cr_2O^{2-}_7}
    \vspace{.35in}
  \end{parts}

  \newpage

  \question Classify as spontaneous or no reaction. If spontaneous, write out the complete reaction.

  \begin{parts}
    \part \chemfig{Ni^{2+}} + \chemfig{Ag{(}s{)}}
    \vspace{.35in}
    \part \chemfig{Zn^{+2}} + Li(s)
    \vspace{.35in}
    \part Ag(s) + \chemfig{l^{-}}
    \vspace{.35in}
    \part \chemfig{H^+} + Cu(s)
    \vspace{.35in}
    \part \chemfig{H^+} + Fe(s)
    \vspace{.35in}
    \part \chemfig{Sn^{4+}} + Au(s)
    \vspace{.35in}
    \part \chemfig{Sn^{2+}} + Co(s)
    \vspace{.35in}
    \part \chemfig{Cu^+} + Sn(s)
    \vspace{.35in}
    \part \chemfig{Al^{3+}} + Ni(s)
    \vspace{.35in}
    \part \chemfig{Hg^{2+}} + \chemfig{Ha_2}(g)
    \vspace{.35in}
  \end{parts}

  \question Which member of each of the following pairs is the stronger oxidizing agent?
  \begin{parts}
    \part \chemfig{Zn^{2+}} or \chemfig{Ca^{2+}}
    \vspace{.35in}
    \part \chemfig{Cr^{3+}} or \chemfig{Cu^{2+}}
    \vspace{.35in}
    \part \chemfig{Br_2} or \chemfig{l_2}
    \vspace{.35in}
  \end{parts}

  \newpage

  \question Which member of each of the following pairs is the stronger reducing agent?
  \begin{parts}

    \part Mn or Pb
    \vspace{.35in}
    \part \chemfig{Cu^+} or \chemfig{Sn^{2+}}
    \vspace{.35in}
    \part \chemfig{Cr^{2+}} or \chemfig{Fe^{2+}}
    \vspace{.35in}

  \end{parts}

  \question Predict whether a spontaneous reaction is expected when the following are mixed, and state the products of any spontaneous reactions.
  \begin{parts}

    \part Zn(s) and \chemfig{H_2}(g)
    \vspace{.35in}
    \part Sn(s) and \chemfig{Sn^{4+}}
    \vspace{.35in}
    \part \chemfig{H^+} is added to Mn(s)
    \vspace{.35in}
    \part \chemfig{Fe^{2+}} is added to \chemfig{Cr_2O^{2-}_7}
    \vspace{.35in}
    \part \chemfig{Fe^{2+}} is added to acidic \chemfig{Cr_2O^{2-}_7}
    \vspace{.35in}
    \part Cu(s) and \chemfig{H^+}
    \vspace{.35in}
    \part A mixture of \chemfig{MnO_2}(s) and \chemfig{H^+} is added to \chemfig{l^{-}}
    \vspace{.35in}
    \part \chemfig{SO^{2-}_4} is added to Sn(s)
    \vspace{.35in}
  \end{parts}

  \newpage

  \question
  \begin{parts}
    \part Which of Cr, \chemfig{l_2}, Al and \chemfig{Fe^3+} will oxidize Co?
    \vspace{.35in}
    \part Which of \chemfig{H_2}, \chemfig{Cl_2}, \chemfig{Hg^{2+}} and \chemfig{H_2O_2} will reduce \chemfig{Ag^+} ?
    \vspace{.35in}
    \part Which of \chemfig{l^{-}}, Pb, \chemfig{Br_2} and \chemfig{Sn^{2+}} will act as reducing agents for \chemfig{Sn^{4+}}?
    \vspace{.35in}
    \part Which of \chemfig{Cu^{2+}}, Zn, acidic \chemfig{NO^{-}_3} and \chemfig{Cl^{-}} will act as oxidizing agents for aqueous
    \vspace{.35in}\chemfig{SO_2} (i.e. \chemfig{H_2SO_3}) ?
    \part Which substance(s) can be oxidized by \chemfig{l_2} but not by acidic \chemfig{SO^{2-}_4}?
    \vspace{.35in}
    \part Which substance(s) can be reduced by \chemfig{l_2} but not by \chemfig{Fe^{2+}}?
    \vspace{.35in}
    \part Which substance(s) can act as an oxidizing agent for Pb but not for \chemfig{Sn^{2+}} ?
    \vspace{.35in}
    \part Which substance(s) will oxidize Co and reduce \chemfig{H^+} ?
    \vspace{.35in}
  \end{parts}

  \question An electrochemical cell was made by joining a half-cell containing 1 M \chemfig{Pb{(}NO_3{)}_2} and a lead electrode to a half-cell consisting of 1 M \chemfig{Zn{(}NO_3{)}_2} and a zinc electrode. As the cell continues to operate, what happens to the [\chemfig{Pb^{2+}}] ? What happens to the [\chemfig{Zn^{2+}}]?

  \newpage

  \question You have been given three half-reactions: \newline
  \schemestart \chemfig{A^{2+}} + 2 \chemfig{e^{-}} \arrow{<=>} A(s) \schemestop \newline
  \schemestart \chemfig{B^{2+}} + 2 \chemfig{e^{-}} \arrow{<=>} B(s) \schemestop \newline
  \schemestart \chemfig{C^{2+}} + 2 \chemfig{e^{-}} \arrow{<=>} C(s). \schemestop \newline
  The reactions are not in any order of tendency to reduce. The following experiment data is found: \newline
  \chemfig{A^{2+}} reacts with C(s) but not with B(s) .
  Arrange the half reactions in decreasing order of tendency to reduce (greatest tendency first).
  \vspace{3in}

  \question You have been given four half-reactions: \newline
  \schemestart \chemfig{D^{2+}} + 2 \chemfig{e^{-}} \arrow{<=>} D(s)  \schemestop \newline
  \schemestart \chemfig{E2^{2+}} + 2 \chemfig{e^{-}} \arrow{<=>} E(s)  \schemestop \newline
  \schemestart \chemfig{F^{2+}} + 2 \chemfig{e^{-}} \arrow{<=>} F(s)  \schemestop \newline
  \schemestart \chemfig{G^{2+}} + 2 \chemfig{e^{-}} \arrow{<=>} G(s).  \schemestop \newline
  Experimentally, it was found that: \newline
  \chemfig{F^{2+}} reacts with D(s), E(s) and G(s) \newline
  no reaction occurs between \chemfig{D^{2+}} and any of the metals \newline
  \chemfig{G^{2+}} only reacts with D(s). \newline
  Arrange the half-reactions decreasing strength as oxidizing agents (greatest strength first).

  \vspace{3in}


  \question You have been given five half-reactions: \newline

  \schemestart \chemfig{H^{2+}} + 2 \chemfig{e^{-}} \arrow{<=>} H(s) \schemestop \newline
  \schemestart \chemfig{I^{2+}} + 2 \chemfig{e^{-}} \arrow{<=>} I(s) \schemestop \newline
  \schemestart \chemfig{J^{2+}} + 2 \chemfig{e^{-}} \arrow{<=>} J(s) \schemestop \newline
  \schemestart \chemfig{K^{2+}} + 2 \chemfig{e^{-}} \arrow{<=>} K(s) \schemestop \newline
  \schemestart \chemfig{L^{2+}} + 2 \chemfig{e^{-}} \arrow{<=>} L(s).  \schemestop \newline
  Experimentally, it was found that: \newline
  \chemfig{K^{2+}} only reacted with I(s) and H(s) \newline
  \chemfig{L^{2+}} did not react with J(s) \newline
  \chemfig{I^{2+}} reacted with H(s). \newline

  Arrange the half-reactions in decreasing tendency to reduce (greatest tendency first).

  \vspace{3in}

  \question Use your Table of Reduction Potentials to complete the following table. Omit the spaces on the diagonal and use "RX" to indicate that a reaction occurs between the metal and ion or use "-' to indicate that no reaction occurs.
  \newline
  \begin{tabular}{|c| c |c| c| c|}
    \hline
    &  \chemfig{Fe^{2+}} &  \chemfig{Au^{3+}} &  \chemfig{Ni^{2+}} &  \chemfig{Pb{^2+}} \\
    \hline
    Fe & & & & \\
    \hline
    Au & & & & \\
    \hline
    Ni & & & & \\
    \hline
    Pb & & & & \\
    \hline
  \end{tabular}

  \vspace{3in}

  \question Given the following data \newline
  \begin{tabular}{|c| c |c| c| c|}
    \hline
    & \chemfig{V^{2+}} & \chemfig{Cd^{2+}} & \chemfig{Ti^{2+}} & \chemfig{Ga^{3+}} \\
    \hline
    V & & Rx & - & Rc \\
    \hline
    Cd & - & & - & - \\
    \hline
    Ti & Rx & Rx & & Rx \\
    \hline
    Ga & - & Rx & - & \\
    \hline
  \end{tabular}
  \newline
  where: "RX" means a reaction occurred and ``-'' means no reaction occurred \newline Arrange the metal ions in decreasing strength as oxidizing agents.


\end{questions}

\end{document}
